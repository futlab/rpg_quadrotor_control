\documentclass[12pt,a4paper,fleqn]{article}

\usepackage[utf8]{inputenc}
\usepackage[english,russian]{babel}

\usepackage{fancyhdr}

\title{A frequency domain итеративный обучающийся алгоритм для высокоточных периодических манёвров квадрокоптера}

\author{
  Markus Hehn, Raffaello D’Andrea
}

\begin{document}

\pagestyle{fancy}             % Fancy headings
\pagenumbering{arabic}				% Begin arabic page numbering (1,2,...)

\maketitle
\tableofcontents
\newpage

\section{Введение} \label{sec:intro}

\section{Обучающийся алгоритм} \label{sec:learn}

В данном разделе описан обучающийся алгоритм, применённый к квадрокоптеру с целью компенсации систематических нарушений, которые ухудшают качество полёта во время выполнения периодических движений. Основная идея --- использовать данные прошлых попыток для определения ошибок следования, и компенсации их при следующих полётах в "беспричинном" виде. Для этой компенсации, мы используем априорное знание преобладающей динамики квадрокоптера при управлении с обратной связью, и совмещаем это знание с данными измерений из экспериментов с целью определить подходящие поправки для применения во время следующей попытки. В сравнении с чистым управлением по обратной связи, эта схема может улучшить точность следования, поскольку повторяющиеся нарушения компенсируются "беспричинно", тогда как управление только по обратной связи ограничено причинными поправками.

Базовая структура системы, используемой обучающимся алгоритмом изображена на Рис. 2. Замкнутая динамика квадрокоптера с управлением по обратной связью остаётся не задетой обучающимся алгоритмом. Мы используем подход подбора уставки контроллера, также известный как последовательная архитектура [24] или непрямое обучающееся управление [23, 43]. Уставка контроллера дополняется добавлением входа поправки $u(t)$. Мы предполагаем, что мы можем вывести(derive) линейную time-invariant(LTI) аппроксимирующую модель замкнутой динамики между ошибкой следования и 

\cite{Faessler17ral}


\newpage
\bibliographystyle{ieeetr}
\bibliography{./library}

\end{document}
